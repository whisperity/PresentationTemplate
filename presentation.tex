\documentclass[aspectratio=169]{beamer}

\usepackage[english]{babel}
\usepackage{booktabs}
\usepackage{caption}
\usepackage{graphicx}
\usepackage{hyperref}
\usepackage{marvosym}
\usepackage{minted}
    \setminted{autogobble}
    \setminted{escapeinside=@@}
    \setminted{mathescape}
\usepackage{multicol,multirow}
\usepackage{svg}
\usepackage{setspace_beamer}
\usepackage{tikz}
    \usetikzlibrary{patterns}
\usepackage[normalem]{ulem}
    \newcommand\csout[1]{\bgroup\markoverwith{\textcolor{#1}{\rule[0.5ex]{4pt}{0.8px}}}\ULon}

\usepackage{blfootnote}
\usepackage{cppligature}

\usepackage{csquotes}
\usepackage[backend=biber,
            bibstyle=numeric,
            citestyle=authortitle-icomp,
            refsection=none,
            url=true]{biblatex}
    % Restore Beamer bibitem icons instead of BiBLaTeX [1], [2], etc.
    \setbeamertemplate{bibliography item}{%
      \ifboolexpr{ test {\ifentrytype{book}} or test {\ifentrytype{mvbook}}
        or test {\ifentrytype{collection}} or test {\ifentrytype{mvcollection}}
        or test {\ifentrytype{reference}} or test {\ifentrytype{mvreference}} }
        {\setbeamertemplate{bibliography item}[book]}
        {\ifentrytype{online}
           {\ifkeyword{gitrepo}
            {\setbeamertemplate{bibliography item}{\includegraphics[width=1.3em,keepaspectratio]{gfx/gitlogo.eps}}}
            {\setbeamertemplate{bibliography item}[online]}}
           {\setbeamertemplate{bibliography item}[article]}}%
      \usebeamertemplate{bibliography item}}

    \defbibenvironment{bibliography}
      {\list{}
         {\settowidth{\labelwidth}{\usebeamertemplate{bibliography item}}%
          \setlength{\leftmargin}{\labelwidth}%
          \setlength{\labelsep}{\biblabelsep}%
          \addtolength{\leftmargin}{\labelsep}%
          \setlength{\itemsep}{\bibitemsep}%
          \setlength{\parsep}{\bibparsep}}}
      {\endlist}
      {\item}
    \addbibresource{bibliography.bib}

\usetheme{focus}

% Show \subsubsections in the ToC and the generated PDF document outline too!
\hypersetup{bookmarksopen=true,bookmarksopenlevel=4}
\setcounter{tocdepth}{4}

% Decrease the number of the current slide so it falls on the same "logical"
% number. Simulates 'allowframebreaks' when not possible (e.g. with \minted).
\newcommand{\decrnumber}{%
  \only<1>{%
    \addtocounter{framenumber}{-1} % Use same count as previous slide.
  }%
}

% Show image wholly on slide with no respect to where the alignment should be.
% (via http://tex.stackexchange.com/a/3927)
\newcommand{\fullslideimage}[1]{%
\begin{tikzpicture}[remember picture, overlay]
    \node[at=(current page.center)]{%
        \includegraphics[keepaspectratio,
                         width=\paperwidth,
                         height=\paperheight]{#1}
    };
\end{tikzpicture}
}
\newcommand{\fullslidecustomimage}[2]{%
\begin{tikzpicture}[remember picture, overlay]
    \node[at=(current page.center)]{%
        \includegraphics[keepaspectratio,
                         width=\paperwidth,
                         height=\paperheight,
                         #2]{#1}
    };
\end{tikzpicture}
}

% Puts an icon, and a text in a set colour into nicely into an
% itemize/enumerate.
\newcommand\icon[3]{%
    $\vcenter{\hbox{\includegraphics[height=1.5em]{gfx/#1}}}$ \textbf{\emph{\color{#2} #3}}%
}
\newcommand\icononly[1]{%
    $\vcenter{\hbox{\includegraphics[height=1.5em]{gfx/#1}}}$%
}

\definecolor{purple}{HTML}{7b68ee}
\definecolor{exampleblue}{HTML}{7289DA}

% C++.
\def\CC{{C\nolinebreak[4]\hspace{-.07em}\raisebox{.1ex}{\small\bf ++}}}


% -----------------------------------------------------------------------------
\title{Title?}
\subtitle{Subtitle}
\author{\texorpdfstring{\uline{Presenter}}{Presenter}}
% \titlegraphic{%
%     \includegraphics[width=5cm]{gfx/logo.png}
% }
\institute{\normalsize Affiliation}
\date{2099.99.99.}

\begin{document}

% Don't show a "Section" slide for any introduction slides. Instead, show a ToC
% for the current section highlighted.
\AtBeginSection[]{
  \begin{frame}[plain,noframenumbering]{}
    \tableofcontents[
      % Show the current 1st level section filled, and grey out the rest.
      sectionstyle=show/shaded,
      % Show the current subsection (which might be not existing), and the rest
      % of the section, hide 2nd levels of other sections.
      subsectionstyle=show/show/hide,
      % Show 3rd level subsubsections of the current subsection (which might
      % not exist), grey the other children of the current 2nd level subsection
      % (once again, usually empty), and show the rest of the current section,
      % but hide children of other 1st levels.
      subsubsectionstyle=show/show/show/hide]
  \end{frame}%
}

% Don't show a "Subsection" heading slide either, instead show a ToC for the
% current subsection.
\AtBeginSubsection[]{
  \begin{frame}[plain,noframenumbering]{}
    \tableofcontents[
      % Show the current 1st level section filled, and grey out the rest.
      sectionstyle=show/shaded,
      % Show the current subsection, grey out the rest of the section,
      % hide 2nd levels of other sections.
      subsectionstyle=show/shaded/hide,
      % Show 3rd level subsubsections of the current subsection, grey the other
      % children of the current 2nd level subsection, and hide children of
      % other 1st levels.
      subsubsectionstyle=show/show/shaded/hide]
  \end{frame}%
}

\begin{frame}
  \maketitle

  {\setstretch{.5}
    {\tiny Sponsor \& grant information.}
  }
\end{frame}

\begin{frame}%[noframenumbering]
  \tableofcontents[subsubsectionstyle=hide]
\end{frame}



% Use a smaller font here so that every reference fits on a slide.
% \AtNextBibliography{\small}
% \begin{frame}[noframenumbering, t,
%     allowframebreaks
%     ]{References}
%     %\nocite{*}
%     \printbibliography
% \end{frame}


\begin{frame}[plain]{Conclusion}
    \begin{tikzpicture}[remember picture, overlay]
        \node[opacity=.66,at=(current page.center)]{%
            \includegraphics[keepaspectratio,
                             width=\paperwidth,
                             height=\paperheight]{example-image-a}
        };
    \end{tikzpicture}
\end{frame}

\end{document}
